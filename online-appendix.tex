% !TeX spellcheck = ru
% !BIB program = biber
\documentclass[11pt]{article}

\begin{filecontents}[overwrite]{online.bib}
@article{inclan1994use,
	title={Use of cumulative sums of squares for retrospective detection of changes of variance},
	author={Inclan, Carla and Tiao, George C},
	journal={Journal of the American Statistical Association},
	volume={89},
	number={427},
	pages={913--923},
	year={1994},
	publisher={Taylor \& Francis Group},
	language = {british}, hyphenation = {british}
}
\end{filecontents}

\usepackage{amsmath}
\usepackage[utf8]{inputenc}
\usepackage[T2A]{fontenc}
\usepackage[russian]{babel}
\usepackage[paper=a4paper]{geometry}
\usepackage{microtype}

\usepackage[
backend=biber,
style=gost-authoryear,
natbib=true,
url=false,
doi=false,
autolang = hyphen
]{biblatex}
% \renewbibmacro{in:}{}
\addbibresource{online.bib}

\setlength{\parindent}{2em}
\DeclareMathOperator*{\argmax}{argmax}
\DeclareMathOperator*{\Cov}{Cov}
\newcommand{\hypo}{\mathcal{H}}
\newcommand{\eqdef}{\mathrel{\stackrel{\text{def}}=}} % Знак «равно по определению»

\title{Онлайн-приложение к~статье \\ «А~был ли~сдвиг: эмпирический анализ тестов на~структурные сдвиги в~волатильности доходностей»}

\author{Андрей Викторович Костырка \and Дмитрий Игоревич Малахов}

\begin{document}

\maketitle

\begin{equation}\label{eq:ait}
	\mathrm{AIT} \eqdef \sup_k \sqrt{C_T \hat\sigma^2 / \hat S} \cdot \left| D_k \right|,
\end{equation}
где $D_k \eqdef C_k / C_T - k / T$, $C_k \eqdef \sum_{t=1}^k \tilde r_t^2$, $C_T \eqdef \sum_{t=1}^T \tilde r_t^2$, $\tilde r_t$ "--- центрированные доходности, $\hat\sigma^2 \eqdef C_T / T$ и~$\hat S$ "--- HAC"=оценка безусловной дисперсии \textit{квадратов} доходностей, т.\,е. $\hat S \eqdef \sum_{j = -m}^m w(j, m) \hat \Omega_j$,где  $\hat\Omega_j = \widehat{\Cov}(\tilde r^2_t, \tilde r^2_{t-j}) \eqdef T^{-1} \sum_{t=j+1}^T (\tilde r^2_t - \hat\sigma^2) (\tilde r^2_{t-j} - \hat\sigma^2)$, $w(j, m)$ "--- ядерные веса выборочных автоковариаций, $m$ "--- ширина окна.

Упрощённая версия данной статистики для IID-рядов:
\begin{equation}
\mathrm{IT} \eqdef \sup_k \sqrt{T/2} \cdot \left|D_k \right|.
\end{equation}


\section{ICSS-алгоритм для выявления структурных сдвигов}

В данном разделе под нулевой гипотезой будет подразумеваться гипотеза $\hypo_0\colon$ отсутствует структурный сдвиг. Порядок проведения данного теста приводится по \citet{inclan1994use}:
\begin{enumerate}
	\item\label{item:one} Получить эмпирические критические значения AIT-статистики из уравнения~\eqref{eq:ait} для ряда с~похожей динамикой и~похожей длиной (с помощью Монте"=Карло"=симуляций) или взять асимптотические критические значения Inclán, Tiao: 1{,}224, 1{,}358, 1{,}628\label{page:critval} для квантилей уровня 90\,\%, 95\,\% и~99\,\% соответственно.
	\item\label{item:two} Посчитать AIT"=статистику для исследуемого ряда.
	\begin{enumerate}
		\item  Если расчётное значение не превышает критического, то остановить тест и~сделать заключение об однородности дисперсии ряда.
		\item\label{item:begin} Если расчётное значение превышает критическое, то зациклить проверку от конца к~началу: определить $k_{\prime} \eqdef \argmax |\{D_k\}_{k=1}^T|$ (момент потенциального структурного сдвига), обрезать выборку \textbf{с~конца} (т.е. оставить наблюдения $\{r_t\}_{t=1}^{k_{\prime}}$), рассчитать AIT"=статистику для $\{r_t\}_{t=1}^{k_{\prime}}$, сравнить с~критическим значением, при превышении оного определить $k_{\prime\prime}\eqdef \argmax_k |\{D_k\}_{k=1}^{k_\prime}|$ и~повторять проверку на укорачиваемой с~конца выборке до тех пор, пока не перестанет отвергаться $\hypo_0$. Определить $k^*_1$ "--- момент первого потенциального сдвига "--- как точку конца минимальной выборки, на которой не отвергается $\hypo_0$, т.\,е. $k_1^* \eqdef k_{\prime\prime\cdots}$ (последний из найденных $k_{\prime}, k_{\prime\prime}, k_{\prime\prime\prime}, \ldots$).
	\end{enumerate}
	\item\label{item:three} Рассмотреть вторую часть выборки, $\{r_t\}_{t=k^*_1+1}^{T}$, рассчитать на ней AIT"=статистику.
	\begin{enumerate}
		\item Если расчётное значение не превышает критического, то заключить, что в~ряде больше сдвигов нет.
		\item\label{item:end} Если расчётное значение превышает критическое, то зациклить проверку от начала обрезанной выборки к~концу: определить $k^{\prime} \eqdef \argmax \allowbreak |\{D_k\}_{k=k^*_1+1}^T|$ (момент потенциального структурного сдвига), обрезать выборку \textbf{с~начала} (оставить наблюдения $\{r_t\}_{t=k^\prime+1}^{T}$), рассчитать AIT"=статистику для $\{r_t\}_{t=k^\prime+1}^{T}$, сравнить с~критическим значением, при превышении оного определить $k^{\prime\prime}\eqdef \argmax_k |\{D_k\}_{k=k^\prime+1}^{T}|$ и~повторять проверку на укорачиваемой с~начала выборке до тех пор, пока не перестанет отвергаться $\hypo_0$. Определить $k^*_2$ "--- момент второго потенциального сдвига "--- как точку, предшествующую точке начала последней укороченной выборки без сдвигов, т.\,е. $k^*_2 \eqdef k^{\prime\prime\cdots}$ (последний из найденных $k^{\prime}, k^{\prime\prime}, k^{\prime\prime\prime}, \ldots$).
	\end{enumerate}
	\item  Если на шагах \ref{item:begin} и~\ref{item:end} были обнаружены моменты сдвигов, то зациклить шаги \ref{item:two} и~\ref{item:three}: рассмотреть укороченную с~обоих концов выборку, т.\,е.\ середину ряда, $\{r_t\}_{t=k^*_1+1}^{k^*_2}$. Повторять шаги \ref{item:two}--\ref{item:three} для этой средней части выборки, находя поочерёдно первый потенциальный сдвиг в~начале и~последний потенциальный сдвиг на оставшейся части и~укорачивая выборку с~обоих концов, пока не перестанет отвергаться нулевая гипотеза о~том, что на оставшейся средней части структурный сдвиг отсутствует, и~сделать вывод о~наличии потенциальных (но ещё не подтверждённых окончательно) упорядоченных по возрастанию моментов сдвига $k^*_{(1)}, \ldots, k^*_{(B)}$ в~динамике параметров, где $B$ "--- общее число выявленных сдвигов.
	\item\label{item:rectify} Уточнить найденные точки потенциальных сдвигов по количеству и~по позиции (так как обычно на предыдущих шагах находится больше точек сдвига, чем есть на самом деле). Определить $k^*_{(0)} \eqdef 0$ и~$k^*_{{B+1}} \eqdef T$. Зациклить следующую процедуру уточнения результатов.
	\begin{enumerate}
		\item\label{item:five} Проверить наличие сдвига на каждом интервале от предшествующего до следующего момента, т.\,е.\ рассчитать AIT-статистики для $\{r_t\}_{t=k^*_{(i-1)}+1}^{k^*_{(i+1)}}$, $i = 1, \ldots, B$.
		\item\label{item:fivethree} Если AIT"=статистики на некоторых интервалах не превышают критического значения и~всего обнаружилось $B' < B$ статистик выше порога, то это значит, что некоторые найденные до этого точки не являются моментами структурного сдвига и~что необходимо исключить их из ранжировки: после всех $B$ проверок исключить из набора $k^*_{(1)} < \ldots < k^*_{(B)}$ точки, для которых не отвергается гипотеза об отсутствии структурного сдвига на интервале от предыдущей до следующей точки, и~переопределить набор потенциальных сдвигов $k^*_{(1)} < \ldots < k^*_{(B')}$.
		\item\label{item:fivetwo} Если все $B$ AIT"=статистик превышают критическое значение, то переопределить $k^*_{(i)}\eqdef \argmax_k |\{D_k\}_{k=k^*_{(i-1)}+1}^{k^*_{(i+1)}}|$, т.\,е.\ уточнить положение каждого потенциального сдвига на интервале. Отсортировать $k^*_{(i)}$ по возрастанию.
		\item Возвращаться на шаг~\ref{item:five}, пока количество потенциальных точек сдвига не перестанет уменьшаться на шаге~\ref{item:fivethree}, а~все значения $k^*_{(i)}$ при отсутствии изменения количества точек не перестанут изменяться с~каждой следующей итерацией более чем на некоторое пороговое значение на шаге~\ref{item:fivetwo} (рекомендуется значение порога в~2~точки).
	\end{enumerate}
\end{enumerate}

На усмотрение исследователя на шаге~\ref{item:rectify} можно проверять дополнительное ограничение: если в~результате очередного изменения некоторое $k^*_{(i)}$ оказалось слишком близко к~$k^*_{(i-1)}$ и~из содержательных соображений нельзя заключить, что структурные сдвиги происходят так часто, то $k^*_{(i)}$ удаляется из набора. Это же ограничение можно применять, если максимальное изменение положения точки шаге~\ref{item:rectify} перестаёт уменьшаться и~остаётся на уровне, выше порогового. В~данной работе используется менее строгое ограничение: минимально допустимое расстояние между $k^*_{(i)}$ и~$k^*_{(i-2)}$ должно быть 20~точек, иначе $k^*_{(i-1)}$~удаляется из набора, а~при наличии зацикливания ограничивается количество итераций (не более~100).

В данной работе мы предлагаем ещё одну проверку, не описанную в~других работах: существует вероятность того, что на шагах \ref{item:two}--\ref{item:three} были выявлены два или более потенциальных сдвигов, однако на шаге~\ref{item:five} ни на одном интервале не обнаруживается значимых сдвигов (этот феномен изредка наблюдался в~симуляциях примерно в~0{,}1\,\%~случаев). Алгоритм предписывает исключить все потенциальные точки сдвига одновременно, что приведёт к~тому, что гипотеза о~наличии в~ряде сдвига отвергается (так как на всём ряде расчётное значение AIT"=статистики превышает критическое), однако достоверно установить присутствие хотя бы одного сдвига не удаётся. В~таком случае мы не исключаем сразу все точки из рассмотрения, а~рассчитываем середины интервалов, определяемых этими точками, и~проводим проверку ещё раз. Например, если $B' = 2$ и~$B'' = 0$, то не исключаются одновременно $k^*_{(1)}$ и~$k^*_{(2)}$, а~задаётся $\tilde B'' = B' - 1 = 1$, рассчитывается $k^{*}_{(\tilde B'')} = (k^*_{(1)} + k^*_{(2)})/2$, и~проверка возобновляется с~шага~\ref{item:fivethree}. Данное решение обусловлено тем фактом, что если на некотором участке ряда на самом деле присутствует один сдвиг, расположенный близко к~краю, то ICSS"=метод, скорее всего, его не обнаружит, и~в случае, когда вместо истинного момента сдвига обнаруживается два близких ложных, расположенных по разные стороны от истинного, скорее всего, на подвыборках, содержащих момент истинного сдвига близко к~краю, никаких сдвигов обнаружено не будет, поэтому при отсутствии добавленной нами проверки оба потенциальных момента сдвига будут ошибочно исключены одновременно.

\printbibliography

\end{document}
